Graph is a very intuitive representation of many datasets,
communities are informally defined as sets of vertexs which are densely connected but scarcely connected to the rest of the  graph,
community detection is an important task that allow  to discovery the structure and organization of online social network.
The problem of community detection has been in intensively researched for more than a half of the century.
In social networks , communities identify groups of users with strong connections which means they may have similiar social relationship.
Several metrics have been proposed as indicators of the quality of a community.
However, social networks, like Wechat, people may have connections online but never meet in the real-life. 
For example, a housing agency comes to your company, and add every one of your collegues as Wechat friend, 
if  only topology structure of graph without information of nodes is used, it is difficult to kick the agency out of your collegue group .
In this paper, we propose a new community detection metric called Location-Based Community Clustering (LBCC) 
which is supposed to detect communities which is connected in real life, and it fulfills a set of minimum structural properties ,which should
be  fulfilled by any community detection metric for social networks, ensure  that the communities are cohesive, structured and well defined.
nodes in a given community generally share attributes or properties
In computational sociology, communities are defined as groups of nodes in social network within which connections are denser than between them. This definition has been found useful also in other type of networks, and community detection became one of the fundamental issues in network science. Community detection has been shown to reveal latent yet meaningful structure not only for groups in online and contact-based social networks,but also in groups of customers with similar interests in inter-disciplinary collaboration networks.
Since in most applications the real communities are not known(often due to the cost of establishing ground truth in large online social netwoks) there is a need for developing reliable metrics to evaluate detected communities, so these metrics can be used to rank the quality of community detection algorithms.Such metrics can also be used to develop novel community algorithms that iteratively attempt to improve the metrics by merging or splitting the given network community structure.

