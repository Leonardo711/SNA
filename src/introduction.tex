For example, a housing agency comes to your company, and add every one of your collegues as Wechat friend, 
if  only topology structure of graph without information of nodes is used, it is difficult to kick the agency out of your collegue group .
Here, location information of users in social networks can be introduced 
and it can help with the main goal of improving the accuracy of detection results based on ground truth, 
i,e, when location information is used,
it is obvious the housing agency doesn't share poi(point of interest) of your company so often, 
while most of your collegues may share frequently.
but it may be asked whether the location information at a set of sample times 
provide enough evidence for reliable community detection,
in other word, 
how much does the observation of locations and colocations of individuals contribute?

In preliminary studies, 
\cite and \cite have studied on relationship between social network structure and geographic distance.
However, those studies do not use location information to improve the results of detection
\cite proposed a community detection method combined network structrue with geographical location.
They redefined every edge as a weighted one and the weights are derived from similarity score 
focused on location-tagged networks, 
they take a place as the location of user home
and based on modularity 

In this paper, we propose a new community detection metric called Location-Based Community Clustering (LBCC) 
which is supposed to detect communities which is connected in real life, and it fulfills a set of minimum structural properties ,which should
be  fulfilled by any community detection metric for social networks, ensure  that the communities are cohesive, structured and well defined.
nodes in a given community generally share attributes or properties
In computational sociology, communities are defined as groups of nodes in social network within which connections are denser than between them. This definition has been found useful also in other type of networks, and community detection became one of the fundamental issues in network science. Community detection has been shown to reveal latent yet meaningful structure not only for groups in online and contact-based social networks,but also in groups of customers with similar interests in inter-disciplinary collaboration networks.
Since in most applications the real communities are not known(often due to the cost of establishing ground truth in large online social netwoks) there is a need for developing reliable metrics to evaluate detected communities, so these metrics can be used to rank the quality of community detection algorithms.Such metrics can also be used to develop novel community algorithms that iteratively attempt to improve the metrics by merging or splitting the given network community structure.

